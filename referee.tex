\documentclass[12pt]{article} 
\usepackage{graphicx}
\usepackage{color}
\def\lsim{\mathrel{\rlap{\lower4pt\hbox{\hskip1pt$\sim$}}
    \raise1pt\hbox{$<$}}}         %less than or approx. symbol
\textwidth=17cm \textheight=22.5cm   
\topmargin -1.5cm \oddsidemargin -0.3cm %\evensidemargin -0.8cm  
\begin{document}
We thank the referee for his statement that our results are useful and
should be published in the literature. While we cannot argue whether
our paper should be published as a letter or not, we take the
referee's two points, the first implicit and the second explicit, 
namely (a) that we should explain more clearly
what is new  in this paper and (b) that we should provide an explicit result
for the cross-section and uncertainty.


Concerning the first point, we have now added to the abstract a sentence
which states clearly that, in comparison to previous work, we have added the
possibility of performing 
matching scale variation, and that we use this for the sake of
comparison to recent work. We have also expanded the second paragraph
of pag.~3, where we elaborate further on the motivation for performing
scale variation studies. Concerning the second point, we have now added
Table~1, where results are given as requested by the referee.

These sentences in the abstract and introduction are meant to provide
an answer to  the referee's main concern that ``the paper does
not really represent any conceptual novelty''. Indeed, while we certainly agree
with him that the methods used  here are the same we previously used
for the case of Higgs production, we do not fully agree with him that
``as a theoretical test case \dots there are no new conclusions
here.'' Indeed the main motivation for performing this study was the
recent papers Ref.~[25] and especially Ref.~[21] in which various
issues related to scale choice in this processes are addressed (and of
which one of us is a co-author). The detailed discussion of scale
dependence which we provide here, which accounts for almost two
thirds of the entire paper, is new. It partly relies on the newly
implemented matching scale variation. and it is meant to directly
address issues raised in these recent papers. In our view, this is the
main novelty in the present study, and thus the main justification for
publishing it. The added sentences are meant to make this clearer to
the reader.

\end{document}
