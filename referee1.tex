\documentclass[12pt]{article} 
\usepackage{graphicx}
\usepackage{color}
\def\lsim{\mathrel{\rlap{\lower4pt\hbox{\hskip1pt$\sim$}}
    \raise1pt\hbox{$<$}}}         %less than or approx. symbol
\textwidth=17cm \textheight=22.5cm   
\topmargin -1.5cm \oddsidemargin -0.3cm %\evensidemargin -0.8cm  
\begin{document}
We thank the referee for his careful reading of our paper and for his
very useful and constructive suggestions.

The main point raised by the referee is the demand that we address the
total $Z$ production cross-section, both by discussing all
contributions in which $b$ quarks appear in the final state in
association with a $Z$, and also by providing predictions for the
total $Z$ cross-section. We agree that both points are interesting,
and we have added the corresponding discussion to the paper.

Specifically we have:
\begin{itemize}
\item added the last sentence of the abstract;
\item added a discussion of the quark-induced contributions in the
  introduction: from  the penultimate paragraph on Pag.~2
  (``Here, the methodology of Refs....''), 
to the last paragraph on Pag.~3
  (``In the 5FS, the $Z$-production...'), including the new Fig.~1
\item added a discussion of the total cross-section, including
  estimates of mass effects on the light-quark induced contribution:
  from the last paragraph on Pag.~12 (``Having determined...'') until
  the first paragraph on Pag.~14 (``In summary....''), including the
  new Fig.~5.
\end{itemize}

We have further addressed each of the individual points raised by the
referee, which we discuss in turn:
\begin{enumerate}
\item we have modified the sentence by removing the reference to
  other matching methods;
\item we have modified the sentence in order to make clear that the
  scale we are talking about is the scale of the process, and that we
  are discussing generic features of matched calculations, not
  specific results of this paper;
\item we added the statement that all results are presented for LHC
  13~TeV (beginning Pag.5);
\item we have added an explanation of why the band is added in the
  beginning of the paragraph before Eq.~(1) (``Because we extend
  plots down...''); why it is appropriate to take Eqs.~(1-2)
  as two extremes, rather than one as the center and the other as a
  variation, sentence two lines after Eq.~(2) (``The two
  options...'') and finally an explanation of the uncertainty band for
  the $\mu_b=2 m_b$ (of which only the upper edge is shown purely for
  reasons of readability of the plot);
\item we have changed the text as suggested by the referee;
\item probably our text was not clearly written, since what we were
  trying to say here was exactly what the referee says (namely that
  the resummed logs are larger than the fixed order ones --- we even
  said this almost verbatim later in the paper: ``raising the matching
  scales...reduces the size of the logs which are resummed''); we have
  changed the wording of the sentence and now state this more
  explicitly.
\end{enumerate}
 
We hope that we have addressed all the issues raised in a satisfactory
way and we believe that the paper has significantly improved as a
consequence of these changes.

\end{document}
